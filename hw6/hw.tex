% Set up the document
\documentclass{article}

% Page size
\usepackage[
    letterpaper,]{geometry}
\usepackage{changepage}

% Lines between paragraphs
\setlength{\parskip}{\baselineskip}
\setlength{\parindent}{0pt}

% Math
\usepackage{mathtools}
\usepackage{amssymb}
\usepackage{amsthm}
\usepackage{commath}

% Number sets
\newcommand{\T}{\mathbb{T}}
\newcommand{\C}{\mathbb{C}}
\newcommand{\N}{\mathbb{N}}
\newcommand{\Q}{\mathbb{Q}}
\newcommand{\R}{\mathbb{R}}
\newcommand{\Z}{\mathbb{Z}}

% Script chars
\newcommand{\D}{\mathcal{D}}

% Links
\usepackage{hyperref}

% Page numbers at top right
\usepackage{fancyhdr}
\pagestyle{fancy}
\fancyhf{}
\fancyhead[R]{\thepage}
\renewcommand\headrulewidth{0pt}

\begin{document}

\textbf{MATH 419 homework 6} \\
\textbf{Matt Wiens \#301294492} \\
\textbf{2020-08-11}

9.32. Verify that $P_j f(x) = \sum_{I \in \D_j} m_I f \chi_I (x)$.

\textit{Solution.}
Fix any $x$, and let $j$ be the unique index such that $x \in I_j$.
Then
%
\begin{equation*}
    \sum_{I \in \D_j} m_I f \chi_I (x)
        = m_{I_j} f
        = \frac{1}{|I_j|} \int_{I_j} f(t) \dif t
        \eqqcolon P_j f(x)
        .
\end{equation*}
%
Hence $P_j f(x) = \sum_{I \in \D_j} m_I f \chi_I (x)$.

\newpage

9.34. Show that $m_I f = (m_{I_l} f + m_{I_r} f) / 2$.

\textit{Solution.}
First, note that
%
\begin{equation*}
    |I_l| = |I_r| = \frac{|I|}{2}
    .
\end{equation*}
%
Using this, we have that
%
\begin{align*}
    m_{I} f
        &\coloneqq \frac{1}{|I|} \int_{I} f(t) \dif t \\
        &= \frac{1}{|I|}
        \del{
        \int_{I_l} f(t) \dif t
        +
        \int_{I_r} f(t) \dif t
        } \\
        &= \frac{1}{|I|} (|I_l| m_{I_l} f + |I_r| m_{I_r} f) \\
        &= \del{\frac{|I_l|}{|I|} m_{I_l} f + \frac{|I_r|}{|I|} m_{I_r} f} \\
        &= \del{\frac{1}{2} m_{I_l} f + \frac{1}{2} m_{I_r} f} \\
        &= \frac{1}{2} \del{m_{I_l} f + m_{I_r} f}
        .
\end{align*}

\newpage

9.37. Use Theorem 9.36 (from the course textbook) to show that
if $f \in L^2(\R)$ is orthogonal to all Haar functions, then $f$
must be zero in $L^2(\R)$.

\textit{Solution.}
Suppose that for all $I \in \D$ we have that $\langle f, h_I \rangle = 0$.
Then we have that for any $x \in \R$,
%
\begin{equation*}
    \sum_{I \in \D} \langle f, h_I \rangle h_I(x) = 0
    .
\end{equation*}
%
From the course textbook we have that this is equivalent to
%
\begin{equation}
    \lim_{N \to \infty} P_N f(x) - \lim_{M \to -\infty} P_M f(x) = 0
    \label{eq:397-i}
    .
\end{equation}
%
From Theorem 9.36, we know that for any $f \in L^2(\R)$,
%
\begin{equation*}
    \lim_{M \to - \infty} \norm{P_M f}_2 = 0,
\end{equation*}
%
so~\eqref{eq:397-i} reduces to
%
\begin{equation}
    \lim_{N \to \infty} P_N f(x) = 0
    \label{eq:397-ii}
    .
\end{equation}
%
Given that Theorem 9.36 also tells us that
%
\begin{equation*}
    \lim_{N \to - \infty} \norm{P_M f - f}_2 = 0,
\end{equation*}
%
using that~\eqref{eq:397-ii} holds for all $x \in \R$, we identify $f$
with $0 \in L^2(\R)$.

\end{document}
