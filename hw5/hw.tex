% Set up the document
\documentclass{article}

% Page size
\usepackage[
    letterpaper,]{geometry}
\usepackage{changepage}

% Lines between paragraphs
\setlength{\parskip}{\baselineskip}
\setlength{\parindent}{0pt}

% Math
\usepackage{mathtools}
\usepackage{amssymb}
\usepackage{amsthm}
\usepackage{commath}

% Number sets
\newcommand{\T}{\mathbb{T}}
\newcommand{\C}{\mathbb{C}}
\newcommand{\N}{\mathbb{N}}
\newcommand{\Q}{\mathbb{Q}}
\newcommand{\R}{\mathbb{R}}
\newcommand{\Z}{\mathbb{Z}}

% Misc sets
\newcommand{\fS}{\mathcal{S}}

% Links
\usepackage{hyperref}

% Page numbers at top right
\usepackage{fancyhdr}
\pagestyle{fancy}
\fancyhf{}
\fancyhead[R]{\thepage}
\renewcommand\headrulewidth{0pt}

\begin{document}

\textbf{MATH 419 homework 5} \\
\textbf{Matt Wiens \#301294492} \\
\textbf{2020-07-28}

7.3. Show that if $f, g \in \fS(\R)$, then the products
$fg, x^kf(x)$ for all $k \geq 0$, and $e^{-2 \pi i x \xi}f(x)$
for all $\xi \in \R$ belong to $\fS(\R)$. Show also that the
derivative $f^{(l)}$ belongs to $\fS(\R)$, for all $l \geq 0$.

\textit{Solution.}
Recall that for a function $f$ to be a member of the Schwartz
class, it must be infinitely differentiable and must satisfy
%
\begin{equation}
    \lim_{|x| \to \infty} |x|^k |f^{(l)}(x)| = 0 \quad \text{for all $k, l \geq 0$}
    .
    \label{eq:73-schwartz}
\end{equation}
%
Let $f, g \in \fS(\R)$. Recall that if we have two functions $f_1, f_2:
\R \to \C$, where $f_1$ is $k$ times differentiable and $f_2$ is $l$
times differentiable, then $f_1 f_2$ is $\min \{k, l\}$ times
differentiable; if both $k, l = \infty$ (that is, both $f_1$ and $f_2$
are infinitely differentiable), then $f_1 f_2$ is infinitely
differentiable. Hence, for any $k \geq 0$, $\xi \in \R$, since
$f, g, x^k, e^{-2 \pi i x \xi}$ are infinitely differentiable,
so are $fg, x^k f(x), e^{- 2 \pi i x \xi} f(x)$.

We will now show that $fg, x^k f(x), e^{- 2 \pi i x \xi} f(x)$
satisfy~\eqref{eq:73-schwartz}. For $fg$, we have for any $n, l \geq 0$,
%
\begin{align*}
    \lim_{|x| \to \infty} |x|^n |(fg)^{(l)}(x)|
        &= \lim_{|x| \to \infty} |x|^n \envert[4]{\sum_{j = 0}^l \binom{l}{j} f^{(l - j)}(x) g^{(l)}(x)} \\
        &\leq \lim_{|x| \to \infty} |x|^n \sum_{j = 0}^l \binom{l}{j} \envert{f^{(l - j)}(x) g^{(l)}(x)} \\
        &= \sum_{j = 0}^l \binom{l}{j} \lim_{|x| \to \infty} |x|^n \envert{f^{(l - j)}(x) g^{(l)}(x)} \\
        &\leq \sum_{j = 0}^l \binom{l}{j} \lim_{|x| \to \infty} |x|^n |x|^n \envert{f^{(l - j)}(x) g^{(l)}(x)} \\
        &= \sum_{j = 0}^l \binom{l}{j} \lim_{|x| \to \infty} \del{|x^|n |f^{(l - j)}(x)|} \del{|x|^n |g^{(l)}(x)|} \\
        &= \sum_{j = 0}^l \binom{l}{j} \del{\lim_{|x| \to \infty} |x|^n |f^{(l - j)}(x)|} \del{\lim_{|x| \to \infty} |x|^n |g^{(l)}(x)|} \\
        &= \sum_{j = 0}^l \binom{l}{j} 0 \cdot 0 \\
        &= 0
    .
\end{align*}
%
In the first line of the above calculation we used the ``general Leibniz
rule;'' in the sixth line we used that we can commute limits with
function multiplication provided that the limit of each function exists.

For $x^k f(x)$ we have for any $n, l \geq 0$,
%
\begin{align*}
    \lim_{|x| \to \infty} |x|^n |(x^k f)^{(l)}(x)|
        &= \lim_{|x| \to \infty} |x|^n \envert[4]{\sum_{j = 0}^l \binom{l}{j} (x^k)^{(j - l)} f^{(l)}(x)} \\
        &\leq \lim_{|x| \to \infty} |x|^n \sum_{j = 0}^l \binom{l}{j} \envert{(x^k)^{(j - l)} f^{(l)}(x)} \\
        &= \sum_{j = 0}^l \binom{l}{j} \lim_{|x| \to \infty} |x|^n \envert{(x^k)^{(j - l)} f^{(l)}(x)} \\
        &= \sum_{j = 0}^l \binom{l}{j} \lim_{|x| \to \infty} |x|^n \envert{C_j x^{k - (j - l)} f^{(l)}(x)} \\
        &= \sum_{j = 0}^l \binom{l}{j} \lim_{|x| \to \infty} C_j |x|^{n + k - (j - l)} |f^{(l)}(x)|
        ,
\end{align*}
%
where
%
\begin{equation*}
    C_j =
    \begin{dcases}
        \frac{k!}{(k - (j - l))!}, & j - l \leq k, \\
        0, & j - l > k.
    \end{dcases}
\end{equation*}
%
Now, for each limit
%
\begin{equation*}
    \lim_{|x| \to \infty} C_j |x|^{n + k - (j - l)} |f^{(l)}(x)|
    ,
\end{equation*}
%
if $j - l > k$ then $C_j = 0$ and hence the limit is zero. If
$j - l \leq k$, then $k - (j - l) \geq 0$ and hence $n + k - (j - l) \geq 0$;
because $f$ is in the Schwartz class, the limit must also be zero in this case.
Therefore the limit is zero for all $j$ and hence, referring back to our previous
calculation, we have
%
\begin{equation*}
    \lim_{|x| \to \infty} |x|^n |(x^k f)^{(l)}(x)| = 0
    .
\end{equation*}

For $e^{- 2 \pi i x \xi} f(x)$ we have for any $n, l \geq 0$,
%
\begin{align*}
    \lim_{|x| \to \infty} |x|^n |(e^{- 2 \pi i x \xi} f)^{(l)}(x)|
        &= \lim_{|x| \to \infty} |x|^n \envert[4]{\sum_{j = 0}^l \binom{l}{j} (e^{- 2 \pi i x \xi})^{(j - l)} f^{(l)}(x)} \\
        &= \lim_{|x| \to \infty} |x|^n \envert[4]{\sum_{j = 0}^l \binom{l}{j} (- 2 \pi i x \xi)^{j - l} e^{- 2 \pi i x \xi}) f^{(l)}(x)} \\
\end{align*}

FIX J - L -> L - J

\newpage

7.13. Verify properties (a)--(g) in Table 7.1 of the course textbook.

\textit{Solution.}

\newpage

9.16. Use the time-frequency dictionary (Table 7.1 of the course textbook)
to find $\widehat{\psi_{j,k}}(\xi)$, for $\psi \in L^2(\R)$.

\textit{Solution.}

\newpage

9.20. Show that $\{h_{j,k}\}_{j,k \in \Z}$ is an orthonormal set; that is,
verify that $\langle h_{j,k}, h_{j^\prime,k^\prime} \rangle = 1$ if
$j = j^\prime$ and $k = k^\prime$, and
$\langle h_{j,k}, h_{j^\prime,k^\prime} \rangle = 0$ otherwise. First show that
the functions $h_{j,k}$ have zero integral: $\int h_{j,k} = 0$.

\textit{Solution.}

\end{document}
