% Set up the document
\documentclass{article}

% Page size
\usepackage[
    letterpaper,]{geometry}
\usepackage{changepage}

% Lines between paragraphs
\setlength{\parskip}{\baselineskip}
\setlength{\parindent}{0pt}

% Math
\usepackage{mathtools}
\usepackage{amssymb}
\usepackage{amsthm}
\usepackage{commath}

% Number sets
\newcommand{\T}{\mathbb{T}}
\newcommand{\C}{\mathbb{C}}
\newcommand{\N}{\mathbb{N}}
\newcommand{\Q}{\mathbb{Q}}
\newcommand{\R}{\mathbb{R}}
\newcommand{\Z}{\mathbb{Z}}

% Misc sets
\newcommand{\fS}{\mathcal{S}}

% Operator
\newcommand{\F}{\mathcal{F}}
\DeclareMathOperator{\supp}{supp}
\DeclareMathOperator{\length}{length}

% convenience
\newcommand{\jp}{j^\prime}
\newcommand{\kp}{k^\prime}

% Links
\usepackage{hyperref}

% Page numbers at top right
\usepackage{fancyhdr}
\pagestyle{fancy}
\fancyhf{}
\fancyhead[R]{\thepage}
\renewcommand\headrulewidth{0pt}

\begin{document}

\textbf{MATH 419 homework 5} \\
\textbf{Matt Wiens \#301294492} \\
\textbf{2020-07-28}

7.3. Show that if $f, g \in \fS(\R)$, then the products
$fg, x^kf(x)$ for all $k \geq 0$, and $e^{-2 \pi i x \xi}f(x)$
for all $\xi \in \R$ belong to $\fS(\R)$. Show also that the
derivative $f^{(l)}$ belongs to $\fS(\R)$, for all $l \geq 0$.

\textit{Solution.}
Recall that for a function $f$ to be a member of the Schwartz
class, it must be infinitely differentiable and satisfy
%
\begin{equation}
    \lim_{|x| \to \infty} |x|^k |f^{(l)}(x)| = 0 \quad \text{for all $k, l \geq 0$}
    .
    \label{eq:73-schwartz}
\end{equation}
%
Let $f, g \in \fS(\R)$. Recall that if we have two functions $f_1, f_2:
\R \to \C$, where $f_1$ is $k$ times differentiable and $f_2$ is $l$
times differentiable, then $f_1 f_2$ is $\min \{k, l\}$ times
differentiable; if both $k, l = \infty$ (that is, both $f_1$ and $f_2$
are infinitely differentiable), then $f_1 f_2$ is infinitely
differentiable. Hence, for any $k \geq 0$, $\xi \in \R$, since
$f, g, x^k, e^{-2 \pi i x \xi}$ are infinitely differentiable,
so are $fg, x^k f(x), e^{- 2 \pi i x \xi} f(x)$.

We will now show that $fg, x^k f(x), e^{- 2 \pi i x \xi} f(x)$
satisfy~\eqref{eq:73-schwartz}. For $fg$, we have for any $n, l \geq 0$,
%
\begin{align*}
    \lim_{|x| \to \infty} |x|^n |(fg)^{(l)}(x)|
        &= \lim_{|x| \to \infty} |x|^n \envert[4]{\sum_{j = 0}^l \binom{l}{j} f^{(l - j)}(x) g^{(j)}(x)} \\
        &\leq \lim_{|x| \to \infty} |x|^n \sum_{j = 0}^l \binom{l}{j} \envert{f^{(l - j)}(x) g^{(j)}(x)} \\
        &= \sum_{j = 0}^l \binom{l}{j} \lim_{|x| \to \infty} |x|^n \envert{f^{(l - j)}(x) g^{(j)}(x)} \\
        &\leq \sum_{j = 0}^l \binom{l}{j} \lim_{|x| \to \infty} |x|^n |x|^n \envert{f^{(l - j)}(x) g^{(j)}(x)} \\
        &= \sum_{j = 0}^l \binom{l}{j} \lim_{|x| \to \infty} \del{|x|^n |f^{(l - j)}(x)|} \del{|x|^n |g^{(j)}(x)|} \\
        &= \sum_{j = 0}^l \binom{l}{j} \del{\lim_{|x| \to \infty} |x|^n |f^{(l - j)}(x)|} \del{\lim_{|x| \to \infty} |x|^n |g^{(j)}(x)|} \\
        &= \sum_{j = 0}^l \binom{l}{j} 0 \cdot 0 \\
        &= 0
    .
\end{align*}
%
In the first line of the above calculation we used the ``general Leibniz
rule;'' in the sixth line we used that we can commute limits with
function multiplication provided that the limit of each function exists.

For $x^k f(x)$ we have for any $n, l \geq 0$,
%
\begin{align*}
    \lim_{|x| \to \infty} |x|^n |(x^k f)^{(l)}(x)|
        &= \lim_{|x| \to \infty} |x|^n \envert[4]{\sum_{j = 0}^l \binom{l}{j} (x^k)^{(l - j)} f^{(j)}(x)} \\
        &\leq \lim_{|x| \to \infty} |x|^n \sum_{j = 0}^l \binom{l}{j} \envert{(x^k)^{(l - j)} f^{(j)}(x)} \\
        &= \sum_{j = 0}^l \binom{l}{j} \lim_{|x| \to \infty} |x|^n \envert{(x^k)^{(l - j)} f^{(j)}(x)} \\
        &= \sum_{j = 0}^l \binom{l}{j} \lim_{|x| \to \infty} |x|^n \envert{C_j x^{k - (l - j)} f^{(j)}(x)} \\
        &= \sum_{j = 0}^l \binom{l}{j} \lim_{|x| \to \infty} C_j |x|^{n + k - (l - j)} |f^{(j)}(x)|
        ,
\end{align*}
%
where
%
\begin{equation*}
    C_j =
    \begin{dcases}
        \frac{k!}{(k - (l - j))!}, & l - j \leq k, \\
        0, & l - j > k.
    \end{dcases}
\end{equation*}
%
Now, for each limit
%
\begin{equation*}
    \lim_{|x| \to \infty} C_j |x|^{n + k - (l - j)} |f^{(j)}(x)|
    ,
\end{equation*}
%
if $l - j > k$ then $C_j = 0$ and hence the limit is zero. If
$l - j \leq k$, then $k - (l - j) \geq 0$ and hence $n + k - (l - j) \geq 0$;
because $f$ is in the Schwartz class, the limit must also be zero in this case.
Therefore the limit is zero for all $j$ and hence, referring back to our previous
calculation, we have
%
\begin{equation*}
    \lim_{|x| \to \infty} |x|^n |(x^k f)^{(l)}(x)| = 0
    .
\end{equation*}

For $e^{- 2 \pi i x \xi} f(x)$ we have for any $n, l \geq 0$,
%
\begin{align*}
    \lim_{|x| \to \infty} |x|^n |(e^{- 2 \pi i x \xi} f)^{(l)}(x)|
        &= \lim_{|x| \to \infty} |x|^n \envert[4]{\sum_{j = 0}^l \binom{l}{j} (e^{- 2 \pi i x \xi})^{(l - j)} f^{(j)}(x)} \\
        &\leq \lim_{|x| \to \infty} |x|^n \sum_{j = 0}^l \binom{l}{j} \envert{(e^{- 2 \pi i x \xi})^{(l - j)} f^{(j)}(x)} \\
        &= \sum_{j = 0}^l \binom{l}{j} \lim_{|x| \to \infty} |x|^n \envert{(e^{- 2 \pi i x \xi})^{(l - j)} f^{(j)}(x)} \\
        &= \sum_{j = 0}^l \binom{l}{j} \lim_{|x| \to \infty} |x|^n \envert{(- 2 \pi i x \xi)^{l - j} e^{- 2 \pi i x \xi} f^{(j)}(x)} \\
        &= \sum_{j = 0}^l \binom{l}{j} (- 2 \pi i \xi)^{l - j} \lim_{|x| \to \infty} |x|^{n + l - j} |f^{(j)}(x)| \\
        &= \sum_{j = 0}^l \binom{l}{j} (- 2 \pi i \xi)^{l - j} \cdot 0 \\
        &= 0
        .
\end{align*}
%
To summarize what we've done so far, since we have shown that
$fg, x^k f(x), e^{- 2 \pi i x \xi} f(x)$ are infinitely differentiable
and satisfy~\eqref{eq:73-schwartz}, we have shown that they are members
of $\fS(\R)$.

Lastly, we need to show that if $f \in \fS(\R)$ then the derivative
$f^{(l)}$ also belongs to $\fS(\R)$, for all $l \geq 0$. That $f^{(l)}$
for all $l \geq 0$ is infinitely differentiable is trivial. To show that
$f^{(l)}$ satisfies~\eqref{eq:73-schwartz}, we have for all
$n, k \geq 0$
%
\begin{equation*}
    \lim_{|x| \to \infty} |x|^n |(f^{(l)})^{(k)}(x)|
    = \lim_{|x| \to \infty} |x|^n |f^{(l + k)}(x)|
    = 0
    .
\end{equation*}
%
Thus $f^{(l)} \in \fS(\R)$ for all $l \geq 0$.


\newpage

7.13. Verify properties (a)--(g) in Table 7.1 of the course textbook.

\textit{Solution.}
For this problem, fix any $f, g \in \fS(\R)$, $a, b \in \C$, and
$h, s \in \R$. We first show that the Fourier transform on $\fS(\R)$ is
linear (hereafter we will refer to the Fourier transform on
$\fS(\R)$ as the ``Fourier transform'' for conciseness):
%
\begin{align*}
    \widehat{af + b g}(\xi)
        &= \int_{\R} (af + b g)(x) e^{- 2 \pi i \xi x} \dif x \\
        &=
        a \int_{\R} f(x) e^{- 2 \pi i \xi x} \dif x
        +
        b \int_{\R} g(x) e^{- 2 \pi i \xi x} \dif x
        \\
        &= a \widehat{f}(\xi) + b \widehat{g}(\xi)
        .
\end{align*}
%
Now we'll show that the Fourier transform turns translation in the
spatial domain into modulation in the frequency domain:
%
\begin{align*}
    \widehat{\tau_h f}(\xi)
        &= \int_{\R} \tau_h f(x) e^{- 2 \pi i \xi x} \dif x \\
        &= \int_{\R} f(x - h) e^{- 2 \pi i \xi x} \dif x \\
        &= \int_{\R} f(y) e^{- 2 \pi i \xi (y + h)} \dif y \\
        &= e^{2 \pi i (-h) \xi} \int_{\R} f(y) e^{- 2 \pi i \xi y} \dif y \\
        &= e^{2 \pi i (-h) \xi} \widehat{f}(\xi) \\
        &= M_{-h} \widehat{f}(\xi)
        ,
\end{align*}
%
where in the third line we used the change of variable $y = x - h$.

Now we'll show that the Fourier transform turns modulation in the
spatial domain into translation in the frequency domain:
%
\begin{align*}
    \widehat{M_h f}(\xi)
        &= \int_{\R} M_h f(x) e^{- 2 \pi i \xi x} \dif x \\
        &= \int_{\R} e^{2 \pi i h x} f(x) e^{- 2 \pi i \xi x} \dif x \\
        &= \int_{\R} f(x) e^{- 2 \pi i (\xi - h) x} \dif x \\
        &= \widehat{f}(\xi - h) \\
        &= \tau_h \widehat{f}(\xi)
        .
\end{align*}
%
Next we'll show that the Fourier transform turns dilation in the
spatial domain into ``inverse dilation'' in the frequency domain:
%
\begin{align*}
    \widehat{D_s f}(\xi)
        &= \int_{\R} D_s f(x) e^{- 2 \pi i \xi x} \dif x \\
        &= \int_{\R} s f(s x) e^{- 2 \pi i \xi x} \dif x \\
        &= \int_{\R} f(y) e^{- 2 \pi i \frac{\xi}{s} y} \dif y \\
        &= \widehat{f}(s^{-1} \xi) \\
        &= s D_{s^{-1}} \widehat{f}(\xi)
        ,
\end{align*}
%
where in the third line we used the change of variable $y = s x$.

Now we'll show that the Fourier transform commutes with reflection:
%
\begin{align*}
    \widehat{\widetilde{f}}(\xi)
        &= \int_{\R} \widetilde{f}(x) e^{- 2 \pi i \xi x} \dif x \\
        &= \int_{\R} f(-x) e^{- 2 \pi i \xi x} \dif x \\
        &= \int_{\R} f(y) e^{- 2 \pi i (- \xi) y} \dif y \\
        &= \widehat{f}(- \xi) \\
        &= \widetilde{\widehat{f}}(\xi)
        ,
\end{align*}
%
where in the third line we used the change of variable $y = - x$.

Next we'll show that the Fourier transform turns conjugation in the
spatial domain into conjugate reflection in the frequency domain:
%
\begin{align*}
    \widehat{\overline{f}}(\xi)
        &= \int_{\R} \overline{f}(x) e^{- 2 \pi i \xi x} \dif x \\
        &= \int_{\R} \overline{f(x)} e^{- 2 \pi i \xi x} \dif x \\
        &= \overline{\int_{\R} f(x) \overline{e^{- 2 \pi i \xi x}} \dif x} \\
        &= \overline{\int_{\R} f(x) e^{- 2 \pi i (- \xi) x} \dif x} \\
        &= \overline{\widehat{f}(-\xi)}
        .
\end{align*}
%
Finally, we show that the Fourier transform turns differentiation in
the spatial domain into polynomial multiplication in the frequency domain:
%
\begin{align*}
    \widehat{f^\prime}(\xi)
        &= \int_{\R} f^\prime(x) e^{- 2 \pi i \xi x} \dif x \\
        &=
        \lim_{l \to \infty}
        \del{
        f(l) e^{- 2 \pi i \xi l}
        -
        f(-l) e^{- 2 \pi i \xi (-l)}
        }
        - (- 2 \pi i \xi) \int_{\R} f(x) e^{- 2 \pi i \xi x} \dif x \\
        &= 2 \pi i \xi \int_{\R} f(x) e^{- 2 \pi i \xi x} \dif x \\
        &= 2 \pi i \xi \widehat{f}(\xi)
        ,
\end{align*}
%
where in the second line we used integration by parts, and in the third
line used that Schwartz class functions decay to zero as their argument
goes to infinity and that complex exponentials are bounded.

\newpage

9.16. Use the time-frequency dictionary (Table 7.1 of the course textbook)
to find $\widehat{\psi_{j,k}}(\xi)$, for $\psi \in L^2(\R)$.

\textit{Solution.}
For this problem we'll use the notation $\F \cbr{f} \coloneqq \widehat{f}$. Fix
any $\psi \in L^2(\R)$. Using Table 7.1 (which was shown in the context
of the Schwartz class, but the properties we'll use here work for
$L^2(\R)$ as well), we have
%
\begin{align*}
    \F \cbr{\psi_{j,k}}(\xi)
        &= \F \cbr{2^{j / 2} \psi(2^jx - k)}(\xi) \\
        &= 2^{j / 2} \F \cbr{\psi(2^jx - k)}(\xi) \\
        &= 2^{j / 2} \F \cbr{\tau_k \psi(2^jx)}(\xi) \\
        &= 2^{j / 2} e^{- 2 \pi i k \xi} \F \cbr{\psi(2^jx)}(\xi) \\
        &= 2^{j / 2} e^{- 2 \pi i k \xi} \F \cbr{2^{-j} D_{2^j} \psi(x)}(\xi) \\
        &= 2^{- j / 2} e^{- 2 \pi i k \xi} \F \cbr{D_{2^j} \psi(x)}(\xi) \\
        &= 2^{- j / 2} e^{- 2 \pi i k \xi} \F \cbr{\psi}(2^{-j} \xi)
        .
\end{align*}

\newpage

9.20. Show that $\{h_{j,k}\}_{j,k \in \Z}$ is an orthonormal set; that is,
verify that $\langle h_{j,k}, h_{j^\prime,k^\prime} \rangle = 1$ if
$j = j^\prime$ and $k = k^\prime$, and
$\langle h_{j,k}, h_{j^\prime,k^\prime} \rangle = 0$ otherwise. First show that
the functions $h_{j,k}$ have zero integral: $\int h_{j,k} = 0$.

\textit{Solution.}
Recall that the Haar wavelet $h(x)$ is given by
%
\begin{equation}
    h(x) \coloneqq - \chi_{[0, 1/2)}(x) + \chi_{[1/2, 1)}(x)
    \label{eq:920-mother}
    .
\end{equation}
%
We'll first show that the functions $h_{j,k}$ have zero integral:
%
\begin{align*}
    \int_{\R} h_{j,k}(x) \dif x
        &= \int_{\R} 2^{j / 2} h(2^j x - k) \dif x \\
        &= 2^{j / 2} \int_{\R} (- \chi_{[0, 1/2)} + \chi_{[1/2, 1)}) (2^j x - k) \dif x \\
        &= 2^{- j / 2} \int_{\R} (- \chi_{[0, 1/2)} + \chi_{[1/2, 1)}) (y) \dif y \\
        &= 2^{- j / 2} \del{- \int_0^{1/2} \dif y + \int_{1/2}^1 \dif y} \\
        &= 0
        ,
\end{align*}
%
where in the third line we used the change of variable $y = 2^j x - k$.

Now, let $j, k, \jp, \kp \in \Z$. We want to show that
%
\begin{equation}
    \langle h_{j, k}, h_{\jp, \kp} \rangle = \delta_{j, \jp} \delta_{k, \kp}
    \label{eq:920-goal}
    .
\end{equation}
%
Noting that
%
\begin{equation}
    \chi_{[a, b)}(2^j x - k)
        = \chi_{[2^{-j} (k + a), 2^{-j}(k + b))}(x)
        \label{eq:920-fuck}
        ,
\end{equation}
%
we see that
%
\begin{equation*}
    \supp(h_{j, k}) = [2^{-j} k, 2^{-j}(k + 1)) \eqqcolon I_{j, k}
    .
\end{equation*}
%
Thus, for $j = \jp$, $k = \kp$, we have
%
\begin{equation*}
    \langle h_{j, k}, h_{\jp, \kp} \rangle
        = \int_{\R} h_{j, k}^2 (x) \dif x
        = 2^j \int_{I_{j, k}} (\pm 1)^2 \dif x
        = 2^j \cdot 2^{-j}
        = 1
        .
\end{equation*}
%
We now consider the cases when $k \neq \kp$. If $j = \jp$, then
$\supp(h_{j, k}) \cap \supp(h_{\jp, \kp}) = \emptyset$ (the intervals
have the same width, but they are disjoint) and so
%
\begin{equation*}
    \langle h_{j, k}, h_{\jp, \kp} \rangle
    = \int_{\R} h_{j, k}(x) h_{\jp, \kp}(x) \dif x
        = 0
        .
\end{equation*}
%
If $j \neq \jp$, then without loss of generality take $j > \jp$.
We either have
%
\begin{equation*}
    \supp(h_{j, k}) \cap \supp(h_{\jp, \kp}) = \emptyset
    ,
\end{equation*}
%
which, again, trivially implies $\langle h_{j, k}, h_{\jp, \kp} \rangle = 0$,
or we have
%
\begin{equation*}
    \supp(h_{j, k}) \subset \supp(h_{\jp, \kp})
    .
\end{equation*}
%
Note that the support of $h_{j, k}$ is at least half the size of the support
of $h_{\jp, \kp}$, and, furthermore, must lie in either the ``negative side''
of the Haar wave or the ``positive side'' (this can be seen by
examining~\eqref{eq:920-mother} together with~\eqref{eq:920-fuck}).
Thus we have, using our earlier result that $\int h_{j, k} = 0$,
%
\begin{equation*}
    \langle h_{j, k}, h_{\jp, \kp} \rangle
    = \int_{\R} h_{j, k}(x) h_{\jp, \kp}(x) \dif x
    = \pm \int_{I_{j, k}} h_{j, k}(x) \dif x
    = \pm \int_{\R} h_{j, k}(x) \dif x
    = 0
    .
\end{equation*}
%
Having now considered all possible cases, we have proven~\eqref{eq:920-goal}.

\end{document}
