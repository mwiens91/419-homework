% Set up the document
\documentclass{article}

% Page size
\usepackage[
    letterpaper,]{geometry}
\usepackage{changepage}

% Lines between paragraphs
\setlength{\parskip}{\baselineskip}
\setlength{\parindent}{0pt}

% Math
\usepackage{mathtools}
\usepackage{amssymb}
\usepackage{amsthm}
\usepackage{commath}

% Number sets
\newcommand{\T}{\mathbb{T}}
\newcommand{\C}{\mathbb{C}}
\newcommand{\N}{\mathbb{N}}
\newcommand{\Q}{\mathbb{Q}}
\newcommand{\R}{\mathbb{R}}
\newcommand{\Z}{\mathbb{Z}}

% Links
\usepackage{hyperref}

% Page numbers at top right
\usepackage{fancyhdr}
\pagestyle{fancy}
\fancyhf{}
\fancyhead[R]{\thepage}
\renewcommand\headrulewidth{0pt}

\begin{document}

\textbf{MATH 419 homework 4} \\
\textbf{Matt Wiens \#301294492} \\
\textbf{2020-07-14}

6.3. Verify that the set $\cbr{e_0, e_1, \ldots, e_{N - 1}}$
is an orthonormal set in $\C^N$.

\textit{Solution.}
To show that this set is orthonormal, fix any
$m, n = 0, \ldots, N - 1$. We need to show that
%
\begin{equation*}
    \langle e_m, e_n \rangle = \delta_{m n}
    .
\end{equation*}
%
Calculating this, we have
%
\begin{align*}
    \langle e_m, e_n \rangle
        &= \sum_{k = 0}^{N - 1} e_m(k) \overline{e_n(k)} \\
        &= \sum_{k = 0}^{N - 1} \frac{1}{\sqrt{N}} \omega^{km} \frac{1}{\sqrt{N}} \overline{\omega^{kn}} \\
        &= \frac{1}{N} \sum_{k = 0}^{N - 1} e^{i 2 \pi k m / N} e^{- i 2 \pi k n / N} \\
        &= \frac{1}{N} \sum_{k = 0}^{N - 1} e^{i 2 \pi k (m - n) / N}
        .
\end{align*}
%
If $m = n$ then
%
\begin{equation*}
    \langle e_m, e_n \rangle
        = \frac{1}{N} \sum_{k = 0}^{N - 1} e^{i 2 \pi k (m - n) / N}
        = \frac{1}{N} \sum_{k = 0}^{N - 1} 1
        = 1
        .
\end{equation*}
%
If $m \neq n$ then we use the fact that the set
%
\begin{equation*}
    \cbr{k (m - n) \bmod N: k = 0, \ldots, N - 1}
    = \cbr{0, \ldots, N - 1}
\end{equation*}
%
which means that
%
\begin{align*}
    \langle e_m, e_n \rangle
        &= \frac{1}{N} \sum_{k = 0}^{N - 1} e^{i 2 \pi k (m - n) / N} \\
        &= \frac{1}{N} \sum_{k = 0}^{N - 1} e^{i 2 \pi k / N} \\
        &= 0
        .
\end{align*}
%
In the last line we used that $e^{i 2 \pi k / N}$, for
$k = 0, \ldots, N - 1$, are the $N$th roots of unity, and hence sum to
zero.

\newpage

6.13. Show that every basis $\cbr{v_1, v_2, \ldots, v_N}$ of
$\C^N$, together with its dual basis $\cbr{w_1, w_2, \ldots, w_N}$,
satisfies the following orthonormality condition:
%
\begin{equation*}
    \langle v_k, w_j \rangle = \delta_{k j}
    .
\end{equation*}
%
Furthermore, show that for every $v \in \C^N$,
%
\begin{equation*}
    v
    = \sum_{j = 1}^N \langle v, w_j \rangle v_j
    = \sum_{j = 1}^N \langle v, v_j \rangle w_j
    .
\end{equation*}

\textit{Solution.}
Following the notation used in the course textbook, let $B$ be the
matrix whose columns are the vectors $v_1, \ldots, v_N$. Then its
inverse $B^{-1}$ is the matrix whose rows are the vectors
$\overline{w_1}, \ldots, \overline{w_N}$. Hence, because $B B^{-1} = I$,
we can read off that for the diagonal entries of the identity matrix, we
have
%
\begin{equation*}
    1 = \sum_{n = 0}^{N - 1} v_k(n) \overline{w_k(n)} = \langle v_k, w_k \rangle
    ;
\end{equation*}
%
and for the off-diagonal entries,
%
\begin{equation*}
    0 = \sum_{n = 0}^{N - 1} v_k(n) \overline{w_j(n)} = \langle v_k, w_j \rangle
    ,
\end{equation*}
%
where $k \neq j$. This shows that $\langle v_k, w_j \rangle = \delta_{k j}$.

Now, let $v \in \C^N$. We need to show that the coefficients of $v$ in
the $\cbr{v_1, \ldots, v_N}$ basis are given by $\langle v, w_j
\rangle$. Let $x$ be the vector containing these coefficients. Then we
have $v = B x$. Left-multiplying by $B^{-1}$ we have $B^{-1} v = x$,
which tells us that
%
\begin{equation*}
    x_j = \sum_{n = 0}^{N - 1} v(n) \overline{w_j(n)} = \langle v, w_j \rangle
    .
\end{equation*}
%
Hence $v = \sum_{j = 1}^N \langle v, w_j \rangle v_j$. Now we'll play a
similar game to determine the coefficients of $v$ in the
$\cbr{w_1, \ldots, w_N}$. Let $y$ be the vector containing these coefficients.
Then we have $v = \overline{(B^{-1})^T} y$. Left multiplying by $\overline{B^T}$,
we get $\overline{B^T} v = y$, and so
%
\begin{equation*}
    y_j = \sum_{n = 0}^{N - 1} v(n) \overline{v_j(n)} = \langle v, v_j \rangle
    .
\end{equation*}
%
Thus we also have $v = \sum_{j = 1}^N \langle v, v_j \rangle w_j$.

\newpage

6.20. Show that if $v, w \in \C^N$, then
$\langle \widehat{v}, \widehat{w} \rangle = \langle v, w \rangle$.

\textit{Solution.}
Let $T = 1 / \sqrt{N} F_N$, and denote $T^*$ its conjugate transpose.
From linear algebra we know for any $v, w \in \C^N$,
$\langle T^* v, w \rangle = \langle v, T w \rangle$, so
using the result from Exercise 6.19 in the course textbook that
$T^* T = I$, we have
%
\begin{equation*}
    \langle v, w \rangle
    = \langle T^* T v, w \rangle
    = \langle T v, T w \rangle
    = \langle \widehat{v}, \widehat{w} \rangle
    .
\end{equation*}

\newpage

6.33. Show that the Discrete Fourier Transform in $\C^N$ of
the Fourier basis vector $e_j$ is given by the standard basis
vector $s_j$; that is, $\widehat{e}_j = s_j$ for $0 \leq j \leq N - 1$.
Start with the case of $N = 4$.

\textit{Solution.}
For any $N$ we have, for $j = 1, \ldots, N$,
%
\begin{equation*}
    \widehat{e_j}(n) = \langle e_j, e_n \rangle = \delta_{j n} = s_j(n)
    .
\end{equation*}
%
Hence $\widehat{e_j} = s_j$. In particular, this also holds for $N = 4$.

\newpage

6.34. Verify that the Discrete Fourier Transform of the
standard basis vector $s_j$ is the complex conjugate of the
Fourier basis vector $e_j$; that is,
$\widehat{s}_j = \overline{e_j}$.

\textit{Solution.}
For any $N$ we have, for $j = 1, \ldots, N$,
%
\begin{equation*}
    \widehat{s_j}(n)
    = \frac{1}{\sqrt{N}} \sum_{k = 0}^{N - 1} \delta_{j k} \omega^{-k n}
    = \frac{1}{\sqrt{N}} \omega^{-j n}
    = \overline{\frac{1}{\sqrt{N}} \omega^{j n}}
    = \overline{e_j}(n)
    .
\end{equation*}
%
Thus, we have that $\widehat{s}_j = \overline{e_j}$.

\end{document}
