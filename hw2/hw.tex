% Set up the document
\documentclass{article}

% Page size
\usepackage[
    letterpaper,]{geometry}

% Lines between paragraphs
\setlength{\parskip}{\baselineskip}
\setlength{\parindent}{0pt}

% Math
\usepackage{mathtools}
\usepackage{amssymb}
\usepackage{amsthm}
\usepackage{commath}

% Number sets
\newcommand{\C}{\mathbb{C}}
\newcommand{\T}{\mathbb{T}}
\newcommand{\N}{\mathbb{N}}
\newcommand{\Q}{\mathbb{Q}}
\newcommand{\R}{\mathbb{R}}
\newcommand{\Z}{\mathbb{Z}}

% Links
\usepackage{hyperref}

% Page numbers at top right
\usepackage{fancyhdr}
\pagestyle{fancy}
\fancyhf{}
\fancyhead[R]{\thepage}
\renewcommand\headrulewidth{0pt}

\begin{document}

\textbf{MATH 419 homework 2} \\
\textbf{Matt Wiens \#301294492} \\
\textbf{2020-06-16}

3.1. Create a MATLAB script to reproduce Figure 3.1 (from the course
  textbook). Experiment with other values of $N$. Describe what you see.
  Near the jump discontinuity you will see an overshoot that remains
  large as $N$ increases. This is the Gibbs phenomenon. Repeat the
  exercise for the square wave function that arises as the periodic
  extension of the following step function $g(\theta)$:
%
\begin{equation*}
    g(\theta) =
    \begin{cases}
        1,&\theta \in [0, \pi) \\
        -1,&\theta \in [- \pi, 0)
    \end{cases}
    .
\end{equation*}

\textit{Solution.}

\newpage

3.11. Prove Lemma 3.10. Give an example to show that if the sequence of
  averages converges, that does not imply that the original sequence
  converges.

\textit{Solution.}

\newpage

3.18. Deduce Corollary 3.17 from Theorem 3.16.

\textit{Solution.}

\newpage

3.21. Show that if $f$ is continuous and $2\pi$-periodic, then it is
  uniformly continuous on $[-\pi, \pi]$. Moreover, the functions
  $g_n(\theta) \coloneqq |f(\theta) - f(\theta + \pi/n)|$ converge
  uniformly to zero.

\textit{Solution.}

\newpage

3.22. A function $f: \T \to \C$ is called Lipchitz if there is a constant $k$ such that
$|f(\theta_1) - f(\theta_2)| \leq k |\theta_1 - \theta_2|$ for all
$\theta_1, \theta_2 \in \T$. Show that if $f$ is Lipshitz, then
%
\begin{equation*}
    \envert{\hat{f}(n)} \leq \frac{C}{n}
\end{equation*}
%
for some constant $C$ independent of $n \in \Z$.

\textit{Solution.}

\newpage

3.26. Show that if $f$ is an integrable even function, then $\hat{f}(n)
  = \hat{f}(-n)$ for all $n \in \Z$. This means that the corresponding
  Fourier series is a purely cosine series expansion.

\textit{Solution.}

\end{document}
