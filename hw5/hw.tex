% Set up the document
\documentclass{article}

% Page size
\usepackage[
    letterpaper,]{geometry}
\usepackage{changepage}

% Lines between paragraphs
\setlength{\parskip}{\baselineskip}
\setlength{\parindent}{0pt}

% Math
\usepackage{mathtools}
\usepackage{amssymb}
\usepackage{amsthm}
\usepackage{commath}

% Number sets
\newcommand{\T}{\mathbb{T}}
\newcommand{\C}{\mathbb{C}}
\newcommand{\N}{\mathbb{N}}
\newcommand{\Q}{\mathbb{Q}}
\newcommand{\R}{\mathbb{R}}
\newcommand{\Z}{\mathbb{Z}}

% Misc sets
\newcommand{\fS}{\mathcal{S}}

% Links
\usepackage{hyperref}

% Page numbers at top right
\usepackage{fancyhdr}
\pagestyle{fancy}
\fancyhf{}
\fancyhead[R]{\thepage}
\renewcommand\headrulewidth{0pt}

\begin{document}

\textbf{MATH 419 homework 5} \\
\textbf{Matt Wiens \#301294492} \\
\textbf{2020-07-28}

7.3. Show that if $f, g \in \fS(\R)$, then the products
$fg, x^kf(x)$ for all $k \geq 0$, and $e^{-2 \pi i x \xi}f(x)$
for all $\xi \in \R$ belong to $\fS(\R)$. Show also that the
derivative $f^{(l)}$ belong to $\fS(\R)$, for all $l \geq 0$.

\textit{Solution.}

\newpage

7.13. Verify properties (a)--(g) in Table 7.1 of the course textbook.

\textit{Solution.}

\newpage

9.16. Use the time-frequency dictionary (Table 7.1 of the course textbook)
to find $\widehat{\psi_{j,k}}(\xi)$, for $\psi \in L^2(\R)$.

\textit{Solution.}

\newpage

9.20. Show that $\{h_{j,k}\}_{j,k \in \Z}$ is an orthonormal set; that is,
verify that $\langle h_{j,k}, h_{j^\prime,k^\prime} \rangle = 1$ if
$j = j^\prime$ and $k = k^\prime$, and
$\langle h_{j,k}, h_{j^\prime,k^\prime} \rangle = 0$ otherwise. First show that
the functions $h_{j,k}$ have zero integral: $\int h_{j,k} = 0$.

\textit{Solution.}

\end{document}
