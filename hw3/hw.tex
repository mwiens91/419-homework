% Set up the document
\documentclass{article}

% Page size
\usepackage[
    letterpaper,]{geometry}
\usepackage{changepage}

% Lines between paragraphs
\setlength{\parskip}{\baselineskip}
\setlength{\parindent}{0pt}

% Math
\usepackage{mathtools}
\usepackage{amssymb}
\usepackage{amsthm}
\usepackage{commath}

% Number sets
\newcommand{\Pc}{\mathcal{P}}
\newcommand{\T}{\mathbb{T}}
\newcommand{\C}{\mathbb{C}}
\newcommand{\N}{\mathbb{N}}
\newcommand{\Q}{\mathbb{Q}}
\newcommand{\R}{\mathbb{R}}
\newcommand{\Z}{\mathbb{Z}}

% Links
\usepackage{hyperref}

% Page numbers at top right
\usepackage{fancyhdr}
\pagestyle{fancy}
\fancyhf{}
\fancyhead[R]{\thepage}
\renewcommand\headrulewidth{0pt}

% Graphics
\usepackage{float}
\usepackage{graphicx}
\graphicspath{ {./img/} }

% MATLAB stuff
\usepackage[numbered,framed]{matlab-prettifier}
\lstset{
  style              = Matlab-editor,
  mlshowsectionrules = true,
}

\begin{document}

\textbf{MATH 419 homework 3} \\
\textbf{Matt Wiens \#301294492} \\
\textbf{2020-06-30}

4.8. Let $f = \chi_{[0, 1]}$. Compute $f * f$ and $f * f * f$ by hand.
Use MATLAB to compute and plot the functions $f$, $f * f$, and
$f * f * f$ over a suitable interval. Notice how the smoothness improves
as we take more convolutions.

\textit{Solution.}
First I will assume that $f: \T \to \C$. We'll start by computing
$f * f$:
%
\begin{align*}
    (f * f)(t)
        &= \frac{1}{2 \pi} \int_{-\pi}^\pi f(y) f(t - y) \dif y \\
        &= \frac{1}{2 \pi} \int_{-\pi}^\pi \chi_{[0, 1]}(y) \chi_{[0, 1]}(t - y) \dif y \\
        &= \frac{1}{2 \pi} \int_0^1 \chi_{[0, 1]}(t - y) \dif y \\
        &= \frac{1}{2 \pi} \int_0^1 \chi_{[0, 1]}(t - y) \dif y \\
    .
\end{align*}
????

\newpage

4.17. Show that if $f$ is integrable and $P \in \Pc_N$, then $f * P \in \Pc_N$.

\textit{Solution.}
If $P \in \Pc_N(\T)$ then we can express
%
\begin{equation*}
    \sum_{|n| \leq N} a_n e^{i n \theta}
\end{equation*}
%
for some set of coefficients $\cbr{a_n}_{n \in \N} \subset \C$. Because
$f$ is integrable we have that
%
\begin{equation*}
    \envert{\int_{-\pi}^\pi f(y) e^{-i n y} \dif y}
    \leq \int_{-\pi}^\pi |f(y)| |e^{-i n y}| \dif y
    = \int_{-\pi}^\pi |f(y)| \dif y
    < \infty
    .
\end{equation*}
%
Define
%
\begin{equation*}
    c_n \coloneqq \frac{1}{2 \pi} a_n \int_{-\pi}^\pi f(y) e^{-i n y} \dif y
    ,
\end{equation*}
%
each of which is finite as shown by our calculation above. Then we have that
%
\begin{align*}
    (f * P) (\theta)
        &= \frac{1}{2 \pi} \int_{-\pi}^\pi f(y) P(\theta - y) \dif y \\
        &= \frac{1}{2 \pi} \int_{-\pi}^\pi f(y) \del{\sum_{|n| \leq N} a_n e^{i n (\theta - y)}} \dif y \\
        &= \frac{1}{2 \pi} \sum_{|n| \leq N} \del{a_n  \int_{-\pi}^\pi f(y) e^{-i n y} \dif y} e^{i n \theta} \\
        &= \sum_{|n| \leq N} c_n e^{i n \theta}
        .
\end{align*}
%
Clearly we have that $f * P \in \Pc_N$.

\newpage

4.22. Suppose $K$ is a continuous function on $\R$ that is zero for all
$|\theta| \geq \pi$. Assume $\int_{- \pi}^\pi K(\theta) \dif \theta = 2 \pi$.
Let $K_n \coloneqq n K(n \theta)$ for $- \pi \leq \theta \leq \pi$. Verify
that $\cbr{K_n}_{n \geq 1}$ is a family of good kernels on $\T$.

\textit{Solution.}
For $\cbr{K_n}_{n \geq 1}$ to be a family of good kernels on $\T$ we require

\begin{adjustwidth}{2.5em}{0pt}
    (a) For all $n \in \N$, $\frac{1}{2 \pi} \int_{- \pi}^\pi K_n(\theta) \dif \theta = 1$;

    (b) There exists $M > 0$ so that
    $\int_{- \pi}^\pi |K_n(\theta)| \dif \theta \leq M$ for all $n \in \N$;

    (c) For each $\delta > 0$,
    $\int_{\delta \leq |\theta| \leq \pi} |K_n(\theta)| \dif \theta \to 0$ as $n \to \infty$.
\end{adjustwidth}

For property (a) we have that
%
\begin{align*}
    \frac{1}{2 \pi} \int_{- \pi}^\pi K_n(\theta) \dif \theta
        &= \frac{1}{2 \pi} \int_{- \pi}^\pi n K(n \theta) \dif \theta \\
        &= \frac{1}{2 \pi} \int_{- n \pi}^{n \pi} K(x) \dif x \\
        &= \frac{1}{2 \pi} \int_{-\pi}^{\pi} K(x) \dif x \\
        &= \frac{1}{2 \pi} \cdot 2 \pi = 1
        ,
\end{align*}
%
where in the second line we used the substitution $x = n \theta$ and in
the third line we used that $K(\theta) = 0$ for $|\theta| \geq \pi$.

For property (b) we use that $K$ is integrable on $[-\pi, \pi]$ (because
it is continuous and the interval is closed) and hence there exists a
constant $M > 0$ such that
%
\begin{equation*}
    \int_{- \pi}^\pi |K(\theta)| \dif \theta < M
    .
\end{equation*}
%
Then we have for all $n \in \N$ that, using the same tricks we used
in proving property (a),
%
\begin{align*}
    \int_{- \pi}^\pi |K_n(\theta)| \dif \theta
        &= \int_{- \pi}^\pi |n K(n \theta)| \dif \theta \\
        &= \int_{- \pi}^\pi n |K(n \theta)| \dif \theta \\
        &= \int_{- n \pi}^{n \pi} |K(x)| \dif x \\
        &= \int_{-\pi}^{\pi} |K(x)| \dif x < M
        ,
\end{align*}
%
which proves property (b).

For property (c), fix any $\delta > 0$. Then we have that
%
\begin{align*}
    \int_{\delta \leq |\theta| \leq \pi} |K_n(\theta)| \dif \theta
        &=
        \int_{- \pi}^{-\delta} |K_n(\theta)| \dif \theta
        +
        \int_{\delta}^{\pi} |K_n(\theta)| \dif \theta
        \\
        &=
        \int_{- \pi}^{-\delta} |n K(n \theta)| \dif \theta
        +
        \int_{\delta}^{\pi} |n K(n \theta)| \dif \theta
        \\
        &=
        \int_{- \pi}^{-\delta} n |K(n \theta)| \dif \theta
        +
        \int_{\delta}^{\pi} n |K(n \theta)| \dif \theta
        \\
        &=
        \int_{- n \pi}^{-n \delta} |K(x)| \dif x
        +
        \int_{n \delta}^{n \pi} |K(x)| \dif x
    .
\end{align*}
%
By taking $n$ sufficiently large, we can guarantee that
%
\begin{equation*}
    \del{[- n \pi, - n \delta] \cup [n \delta, n \pi]} \cap [- \pi, \pi] = \emptyset
    .
\end{equation*}
%
and hence $K$ is zero within our limits of integration above if we take
$n$ large enough (recall that $K$ is zero outside of $[- \pi, \pi]$). Therefore
%
\begin{equation*}
    \int_{\delta \leq |\theta| \leq \pi} |K_n(\theta)| \dif \theta
        =
        \int_{- n \pi}^{-n \delta} |K(x)| \dif x
        +
        \int_{n \delta}^{n \pi} |K(x)| \dif x
        = 0 + 0 = 0
\end{equation*}
%
as $n \to \infty$. This proves property (c).

\newpage

4.31. Show that the Fejér kernel $F_N$ is a good kernel and that
$\sigma_N f = F_N * f$.

\textit{Solution.}
To show that $F_N$ is a good kernel we need to prove the three
properties listed in the solution to the previous question (4.22).

\newpage

5.16. Fix $g \in L^2(\T)$. Suppose that $f, h_k \in L^2(\T)$ for $k \in \Z$,
and that $f = \sum_{k = 1}^\infty h_k$ in the $L^2$ sense. Show that
$\sum_{k = 1}^\infty \langle g, h_k \rangle = \langle g, \sum_{k = 1}^\infty h_k \rangle$.

\textit{Solution.}
Hint: set $f_n = \sum_{k = 1}^n h_k$ in prop 5.15.

\newpage

5.17. Let $\cbr{e_n}_{n \in \Z}$ be the set of trigonometric functions.
Suppose that $\cbr{a_n}_{n \in \Z}$, $\cbr{b_n}_{n \in \Z}$ are sequences
of complex numbers, $f = \sum_{n \in \Z} a_n e_n$, and
$g = \sum_{n \in \Z} b_n e_n$, where the equalities are in the $L^2$ sense.
Show that $\langle f, g \rangle = \sum_{n \in \Z} a_n \overline{b}_n$.
In particular, show that $\norm{f}^2_{L^2(\T)} = \sum_{n \in \Z} |a_n|^2$.

\textit{Solution.}

\end{document}
