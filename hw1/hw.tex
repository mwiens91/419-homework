% Set up the document
\documentclass{article}

% Page size
\usepackage[
    letterpaper,]{geometry}

% Lines between paragraphs
\setlength{\parskip}{\baselineskip}
\setlength{\parindent}{0pt}

% Math
\usepackage{mathtools}
\usepackage{amssymb}
\usepackage{amsthm}
\usepackage{commath}

% Number sets
\newcommand{\C}{\mathcal{C}}
\newcommand{\N}{\mathbb{N}}
\newcommand{\Q}{\mathbb{Q}}
\newcommand{\R}{\mathbb{R}}
\newcommand{\Z}{\mathbb{Z}}

% Links
\usepackage{hyperref}

% Page numbers at top right
\usepackage{fancyhdr}
\pagestyle{fancy}
\fancyhf{}
\fancyhead[R]{\thepage}
\renewcommand\headrulewidth{0pt}

\begin{document}

\textbf{MATH 419 homework 1} \\
\textbf{Matt Wiens \#301294492} \\
\textbf{2020-06-02}

1.2. Write $a_n$ in terms of $b_n$ and $c_n$.

\textit{Solution.}

\newpage

1.7. Verify that if $f$ is a trigonometric polynomial, then its
coefficients $\cbr{a_n}_{n \in \Z}$ are given by
%
\begin{equation*}
    a_n = \frac{1}{2 \pi} \int_{-\pi}^\pi f(\theta) e^{i n \theta} \dif \theta
    .
\end{equation*}

\textit{Solution.}

\newpage

1.14. Use MATLAB to reproduce the plot of the sawtooth function in
  Figure 1.3 (of the course textbook). Then modify your MATLAB code to
  create a plot over the interval $[-3 \pi, 3 \pi)$ for the periodic
  extension of the function defined by $f(\theta) = \theta^2$ for
  $\theta \in [-\pi, \pi)$.

\newpage

1.16. Verify that if $f$ is $2\pi$-periodic and integrable, then
%
\begin{equation*}
    \int_a^{a + 2 \pi} f(\theta) \dif \theta = \int_{-\pi}^\pi f(\theta) \dif \theta
\end{equation*}
%
for all $a \in \R$.

\textit{Solution.}

\newpage

1.17. Verify that for each $n \in \Z$, the function
  $e^{2 \pi i n \theta / L}$ is $L$-periodic.

\textit{Solution.}

\newpage

1.18. Let $f$ be an $L$-periodic trigonometric polynomial of degree $M$;
  that is,
%
\begin{equation*}
    f(\theta) = \sum_{n = - M}^M a_n e^{2 \pi i n \theta / L}
    .
\end{equation*}
%
Verify that $f$ coincides with its $L$-Fourier series.

\textit{Solution.}

\end{document}
